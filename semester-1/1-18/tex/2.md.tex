\section{Отличия языка С++ от языка С}

Язык программирования \textbf{С++} многое, в том числе синтаксис,
унаследовал от \textbf{С}. Обратная совместимость с С также является
одной из целей создателей языка. Однако между этими языками есть и
существенные различия.

\begin{enumerate}
\item
  Различен подход к управлению динамической памятью. В С используются
  \texttt{malloc()} и \texttt{free()}, в С++ - \texttt{new} и
  \texttt{delete} (и их вариации).
\item
  Различны способы представления и работы со строками. В С под строкой
  понимается последовательность (точнее, массив - все элементы
  расположены в смежных ячейках памяти) символов \texttt{char},
  оканчивающихся т. н. \emph{null-терминатором} - символом с кодом
  \emph{0}. В С++ для работы со строками стандартной библиотекой
  предоставлен тип \texttt{std::string}.
\item
  Различны возможности по организации кода. В С++ существует понятие
  \textbf{пространства имен} - это декларативная область, в рамках
  которой определяются различные идентификаторы (имена типов, функций,
  переменных, и т.д.). Пространства имен позволяют предотвращать
  конфликт имен (коллизии) - типы с одинаковым названием, но в разных
  пространствах имен считаются различными и доступ к ним однозначен.
  Помимо этого в С++ можно обращаться к типам структур без использования
  ключевого слова \texttt{struct}, что является \textbf{обязательным} в
  \textbf{С} и создает необходимость использования \texttt{typedef}.
\item
  Для косвенного обращения к данным помимо указателей (переменных,
  хранящих в качестве значения адрес ячейки памяти) в С++ существуют
  ссылки - их использование удобнее, поскольку не требует постоянного
  повторения оператора обращения через указатель (\texttt{*} или
  \texttt{-\textgreater{}}).
\item
  В С++ реализована поддержка \textbf{объектно-ориентированного
  программирования} (ООП), при котором программа рассматривается как
  набор объектов, взаимодействующих друг с другом. Можно определять поля
  и функции (точнее, такие функции называются методы), связанные с
  конкретным объектом (чаще всего это помещается в определении класса
  этого объекта). В тоже время встроенной поддержки ООП в \textbf{С}
  нет.
\item
  С++ позволяет писать более гибкий код с помощью перегрузки функций
  (методов) и операторов.
\item
  В С++ существует механизм обработки ошибок - исключения.
\item
  В современном стандарте С есть ключевые слова, которых нет в С++,
  например \texttt{restrict}, сигнализирующее о единственности указателя
  на заданную область памяти.
\end{enumerate}

Идентификатор --- это последовательность символов, используемая для
обозначения переменной, функции или любого другого объекта. Ключевые
слова - это предварительно определенные зарезервированные
идентификаторы, имеющие специальные значения. Их нельзя использовать в
качестве идентификаторов в программе.
