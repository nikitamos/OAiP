\section{Основные характеристики языка С++}

\textbf{C++} - это компилируемый статически типизированный язык
программирования общего назначения. Язык программирования --- это набор
формальных правил, по которым пишутся программы.

Язык С++ \textbf{является}:
\begin{enumerate}
    \item Компилируемым - программы, написанные на
        С++, перед выполнением сперва преобразуются в целевой (машинный) код
        целевой платформы - компилируется; за это отвечает специальная программа
        - компилятор. В результате получается исполнимый модуль, который уже
    может быть запущен на исполнение как отдельная программа; 
    \item Статически
    типизированным - за каждой переменной закреплен определенный
    \textbf{тип} - класс данных, характеризуемый членами класса и
    операциями, которые могут быть к ним применены. Тип переменной задается
    единожды при ее объявлении и не может быть изменен;
    \item Слабо типизированным - значения разных, порой несвязных, типов в С++ можно
    приводить друг к другу встроенными в язык методами;
    \item Высокоуровневым -
    программы на С++ проще в понимании человеком, чем с программы в машинных
    кодах и на языке ассемблера; \item Мультипарадигмальным - С++ поддерживает
    несколько различных парадигм программирования - совокупностей идей и
    понятий, определяющих стиль написания программ, иными словами, парадигмы
    - подходы к программированию. В частности, С++ поддерживает: -
    Процедурное программирование - парадигма, при которой последовательно
    выполняемые операторы можно собрать в подпрограммы, то есть более
    крупные целостные единицы кода, с помощью механизмов самого языка; -
    Обобщенное программирование - парадигма, заключающаяся в таком описании
    данных и алгоритмов, которое можно применять к различным типам данных,
    не меняя само это описание; - \textbf{Объектно-ориентированное
    программирование} - парадигма, при котором программа рассматривается как
    набор объектов, взаимодействующих друг с другом. У каждого есть свойства
    и поведение;
    \item Языком общего назначения - на С++ пишутся программы для
    различный сфер, начиная встраиваемыми системами и заканчивая разработкой
    игр. В качестве примера можно привести \emph{драйверы} периферийных
    устройств, \emph{операционные системы} и их компоненты, \emph{браузеры},
    \emph{игры} и \emph{игровые движки}, \emph{базы данных}, \emph{системы
    программирования}, в том числе \emph{другие языки программирования} и
    библиотеки для них и т. д.
\end{enumerate}
    
Язык С++ также обладает богатой стандартной библиотекой, включающей в
том числе общеупотребительные структуры данных и алгоритмы.
    
С++ строго регламентирован Международной организации по стандартизации
(ISO). На сегодняшний день выпущен стандарт С++23 и разрабатывается
стандарт С++26.

Комментарии автора

TODO: Возможно, стоит добавить определения альтернативных
подходов/терминов?
