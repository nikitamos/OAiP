\section{Область применения и системы программирования языка С++}

Язык \textbf{С++} получил широкое распространение в сферах с
требованиями к быстродействию ПО, а также в областях, требующих работу с
низкоуровневыми интерфейсами. На \textbf{С++} разрабатывают, в том числе
(\hyperref[examples_3]{примеры тут})
\begin{enumerate}
  \item Драйверы устройств;
  \item Операционные системы и их компоненты;
  \item Базы данных;
  \item Другие языки программирования: компиляторы, интерпретаторы,
  программные библиотеки;
  \item В целом системы программирования: редакторы
  исходного кода, в т.ч. IDE (интегрированная среда разработки),
  отладчики;
  \item Прикладное ПО: браузеры, 3D-редакторы, программы для
  редактирования текста и видео;
  \item Игры и игровые движки;
  \item \ldots{}
\end{enumerate}

Такое разнообразие в первую очередь обусловлено высокому
\emph{быстродействию} и \emph{гибкости} программ, написанных на
\textbf{С++}.

\textbf{Система программирования} - совокупность языка программирования
и программных средств, обеспечивающих подготовку исходного кода
программы, его перевод на машинный код, и последующую отладку. Иными
словами системы программирования создаются для удобства работы
пользователя с выбранным языком программирования.

Как правило, системы программирования включают в свой состав: -
интегрированную среду разработки или программирования (Integrated
Development Environment - IDE); - компилятор; - редактор связей или
компоновщик; - библиотеки заголовочных файлов; - библиотеки классов и
функций; - программы-утилиты.

Наиболее распространены следующие системы программирования (не включая
IDE):

\begin{longtable}[]{@{}
  >{\raggedright\arraybackslash}p{(\columnwidth - 8\tabcolsep) * \real{0.2913}}
  >{\raggedright\arraybackslash}p{(\columnwidth - 8\tabcolsep) * \real{0.0971}}
  >{\raggedright\arraybackslash}p{(\columnwidth - 8\tabcolsep) * \real{0.1068}}
  >{\raggedright\arraybackslash}p{(\columnwidth - 8\tabcolsep) * \real{0.2136}}
  >{\raggedright\arraybackslash}p{(\columnwidth - 8\tabcolsep) * \real{0.2913}}@{}}
\toprule\noalign{}
\begin{minipage}[b]{\linewidth}\raggedright
Набор инструментов (toolchain)
\end{minipage} & \begin{minipage}[b]{\linewidth}\raggedright
Компилятор
\end{minipage} & \begin{minipage}[b]{\linewidth}\raggedright
Компоновщик
\end{minipage} & \begin{minipage}[b]{\linewidth}\raggedright
Стандартная библиотека
\end{minipage} & \begin{minipage}[b]{\linewidth}\raggedright
Отладчик
\end{minipage} \\
\midrule\noalign{}
\endhead
\bottomrule\noalign{}
\endlastfoot
GСС & g++ & ld & libstc++ & gdb \\
LLVM & clang++ & lld & libc++ & lldb \\
MSVC & cl.exe & link.exe & MSVC STL & Visual Studio Windows Debugger \\
\end{longtable}

\textbf{Интегрированную Среду Разработки} можно трактовать как среду в
которой есть все необходимое для проектирования, запуска и тестирования
приложений и где все нацелено на облегчение процесса создания программ.

Что требуется от IDE: - Способность IDE корректно «понимать» код. IDE
должна уметь индексировать все файлы проекта, а также все сторонние и
системные заголовочные файлы и определения (defines, macro). - IDE
должна предоставлять возможность кастомизации команд для построения
проекта, а так же где искать заголовочные файлы и определения. - Должна
эффективно помогать в наборе кода, т.е. предлагать наиболее подходящие
варианты завершения, предупреждать об ошибках синтаксиса и т.д. -
Навигация по большому проекту должна быть удобной, а нахождение
использования быстрым и простым. - Предоставлять широкие возможности для
рефакторинга: переименование и т.д. - Также необходима способность к
генерации шаблонного кода --- создание каркаса нового класса,
заголовочного файла и файла с реализацией. Генерация геттеров/сеттеров,
определения методов, перегрузка виртуальных методов, шаблоны реализации
чисто виртуальных классов (интерфейсов) и т.д.

В качестве примеров можно привести
\begin{enumerate}
  \item Microsoft Visual Studio;
  \item JetBrains CLion;
  \item Qt Creator.
\end{enumerate}

\subsection{Примеры}\label{examples_3}

\begin{enumerate}
\item
  Драйверы в Windows;
\item
  Ядра ОС обычно не пишут на С++, а вот API и драйверы - могут.
  Например, Windows;
\item
  MySQL, свободная реляционная БД;
\item
  LLVM - библиотека и программная платформа для написания языков
  программирования - сама написана на С и С++;
\item
  VisualStudio разрабатывается на C\# и C++;
\item
  Mozilla Firefox и Google Chrome, Blender, LibreOffice, Premiere Pro
\item
  Unreal Engine, Unity, Godot - все это на С++
\item
  \ldots{}
\end{enumerate}
