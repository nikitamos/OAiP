\section{Арифметические и поразрядные операции. Результат
операции}\label{ux430ux440ux438ux444ux43cux435ux442ux438ux447ux435ux441ux43aux438ux435-ux438-ux43fux43eux440ux430ux437ux440ux44fux434ux43dux44bux435-ux43eux43fux435ux440ux430ux446ux438ux438.-ux440ux435ux437ux443ux43bux44cux442ux430ux442-ux43eux43fux435ux440ux430ux446ux438ux438}

Пусть \texttt{a}, \texttt{b} - операнды. К арифметическим операциям в
С++ относятся:

\begin{longtable}[]{@{}lll@{}}
\toprule\noalign{}
Операция & Вызов & Тип операндов \\
\midrule\noalign{}
\endhead
\bottomrule\noalign{}
\endlastfoot
Сложение & \texttt{a\ +\ b} & Целые/вещественные числа \\
Вычитание & \texttt{a\ -\ b} & Целые/вещественные числа \\
Умножение & \texttt{a\ *\ b} & Целые/вещественные числа \\
Деление нацело & \texttt{a\ /\ b} & Целые числа \\
Вещественное деление & \texttt{a\ /\ b} & Вещественные числа \\
Взятие остатка от деления & \texttt{a\ \%\ b} & Целые числа \\
Унарный плюс & \texttt{+a} & Целое/вещественное число \\
Унарный минус & \texttt{-a} & Целое/вещественное число \\
\end{longtable}

Результатом арифметических операций являются числа (целые или
вещественные).

Пусть n - число разрядов (бит) в числах \texttt{a} и \texttt{b}. Пусть
также \(a[i]\) означает \(i\)-ый бит числа a. К поразрядным операциям в
С++ относятся:

\begin{longtable}[]{@{}
  >{\raggedright\arraybackslash}p{(\columnwidth - 6\tabcolsep) * \real{0.0923}}
  >{\raggedright\arraybackslash}p{(\columnwidth - 6\tabcolsep) * \real{0.0295}}
  >{\raggedright\arraybackslash}p{(\columnwidth - 6\tabcolsep) * \real{0.0480}}
  >{\raggedright\arraybackslash}p{(\columnwidth - 6\tabcolsep) * \real{0.8303}}@{}}
\toprule\noalign{}
\begin{minipage}[b]{\linewidth}\raggedright
Операция
\end{minipage} & \begin{minipage}[b]{\linewidth}\raggedright
Вызов
\end{minipage} & \begin{minipage}[b]{\linewidth}\raggedright
Тип операндов
\end{minipage} & \begin{minipage}[b]{\linewidth}\raggedright
Примечание
\end{minipage} \\
\midrule\noalign{}
\endhead
\bottomrule\noalign{}
\endlastfoot
Побитовое И & \texttt{a\ \&\ b} & Целые числа & \texttt{r\ =\ a\ \&\ b;}
\(\forall i=1...n,\ r[i] = a[i] \land b[i]\) \\
Побитовое ИЛИ & \texttt{a\ \textbackslash{}\textbar{}\ b} & Целые числа
& \texttt{r\ =\ a\ \textbackslash{}\textbar{}\ b;}
\(\forall i=1...n,\ r[i] = a[i] \lor b[i]\) \\
Побитовое ИСКЛЮЧАЮЩЕЕ ИЛИ & \texttt{a\ \^{}\ b} & Целые числа &
\texttt{r\ =\ a\ \^{}\ b;}
\(\forall i=1...n,\ r[i] = a[i] \oplus b[i]\) \\
Побитовая инверсия & \texttt{\textasciitilde{}a} & Целое число &
\texttt{r\ =\ \textasciitilde{}a;}
\(\forall i=1...n,\ r[i] = \neg{a[i]}\) \\
Побитовое сдвиг вправо & \texttt{a\ \textless{}\textless{}\ b} & Целые
числа & Сдвигает биты числа \texttt{a} на \texttt{b} разрядов влево.
Если \(b\gt n\) или \(b\lt 0\), результат не определен \\
Побитовый сдвиг влево & \texttt{a\ \textgreater{}\textgreater{}\ b} &
Целые числа & Сдвигает биты числа \texttt{a} на \texttt{b} разрядов
вправо. Если \(b\gt n\) или \(b\lt 0\), результат не определен.Результат
также не определен, если \(a \lt 0\), хотя большинство платформ применят
расширение знаковым битом \\
\end{longtable}

Результатом поразрядных операций являются целые числа.
