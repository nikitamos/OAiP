\section{Символьные и Строковые константы, отличие от числовых констант}

\subsection{Символьные и строковые константы}

\textbf{Константа} - это ограниченная последовательность символов
алфавита языка, представляющая собой изображение фиксированного
(неизменяемого) объекта.

Речь в первую очередь пойдет о \emph{литералах} и их типах.

Символьные литералы представляют собой 1 символ (или часть символа, если
говорить про UTF-*) в какой-либо кодировке. В качестве символьных
констант также могут использоваться управляющие коды, не имеющие
графического представления. При этом код управляющего символа начинается
с символа \texttt{\textbackslash{}}(обратный слеш).

\begin{scriptsize}
\begin{longtable}[]{@{}
  >{\raggedright\arraybackslash}p{(\columnwidth - 4\tabcolsep) * \real{0.0549}}
  >{\raggedright\arraybackslash}p{(\columnwidth - 4\tabcolsep) * \real{0.9066}}
  >{\raggedright\arraybackslash}p{(\columnwidth - 4\tabcolsep) * \real{0.0385}}@{}}
\toprule\noalign{}
\begin{minipage}[b]{\linewidth}\raggedright
Тип
\end{minipage} & \begin{minipage}[b]{\linewidth}\raggedright
Пояснение
\end{minipage} & \begin{minipage}[b]{\linewidth}\raggedright
Пример
\end{minipage} \\
\midrule\noalign{}
\endhead
\bottomrule\noalign{}
\endlastfoot
\texttt{char} & ASCII. Также используется для хранения байт в
многобайтных кодировках &
\texttt{\textquotesingle{}a\textquotesingle{}} \\
\texttt{wchar\_t} & Зависит от платформы, но так или иначе не меньше чем
\texttt{char}. Например, на Linux имеет размер 32 бита и хранит в
кодировке UTF-32, на Windows - 16 бит и хранит UTF-16 &
\texttt{L\textquotesingle{}Р\textquotesingle{}} \\
\texttt{char8\_t} & UTF-8, 8 бит &
\texttt{u8\textquotesingle{}w\textquotesingle{}} \\
\texttt{char16\_t} & UTF-16, 16 бит &
\texttt{u\textquotesingle{}L\textquotesingle{}} \\
\texttt{char32\_t} & UTF-32, 32 бита &
\texttt{U\textquotesingle{}Ё\textquotesingle{}} \\
\end{longtable}
\end{scriptsize}

Строковые литералы представляют собой последовательность символов,
заключенные в кавычки. В памяти строки представляют собой
последовательность символов (точнее, массив, т.к. все символы находятся
друг за другом), ограниченные символом \texttt{\textbackslash{}0}. Этот
нулевой символ называется также \textbf{null-терминатором}, и является
маркером конца строки.

\begin{verbatim}
  const char *example_string = "I am a string";
\end{verbatim}
Вообще говоря строки можно составить из символов любого типа:

\begin{verbatim}
const wchar_t *wide_string = L"Я транслятор";
const char8_t *utf8_string = u8"Judgement!";
const char16_t *utf16_string = u"\";
const char32_t *utf32_string = U"Съешь же ещё этих мягких"
                               U"французских булок, "
                               U"да выпей чаю.";
\end{verbatim}

Однако они всегда будут оканчиваться символом с кодом 0.

\subsection{Отличия}\label{ux43eux442ux43bux438ux447ux438ux44f}

Символьные, строковые и числовые константы - это не одно и тоже.

Числовые константы предназначены для хранения чисел. Символы кодируются
в компьютере с помощью чисел (в соответствии с \textbf{кодовой
таблицей}), однако числа сами по себе не несут какого-либо смысла и не
обозначают печатный или управляющий знак.

Символы и строки также различны:

\begin{verbatim}
char a_char = 'A';
const char *a_string = "A";
\end{verbatim}

Помимо разницы в синтаксисе (разные кавычки), строка \texttt{a\_string}
состоит из двух символов -
\texttt{\textquotesingle{}A\textquotesingle{}} и
\texttt{\textquotesingle{}\textbackslash{}0\textquotesingle{}}.
\texttt{a\_char} и \texttt{a\_string} также занимают разное количество
памяти (\texttt{a\_char} - 1 байт, \texttt{a\_string} - 2 байта).
