\section{Выражения: математические, логические,
текстовые}\label{ux432ux44bux440ux430ux436ux435ux43dux438ux44f-ux43cux430ux442ux435ux43cux430ux442ux438ux447ux435ux441ux43aux438ux435-ux43bux43eux433ux438ux447ux435ux441ux43aux438ux435-ux442ux435ux43aux441ux442ux43eux432ux44bux435}

\textbf{Операнд} - объект над которым происходит операция.

\textbf{Выражение} - комбинация знаков операций и операндов, результатом
которой является определенное значение. Знаки операций определяют
действия, которые должны быть выполнены над операндами. Каждый операнд в
выражении может быть выражением. Значение выражения зависит от
расположения знаков операций и круглых скобок в выражении, а также от
приоритета выполнения операций.

Каждое выражение в С++ характеризуется своим типом. Результирующий тип
выражения выводится из типов операндов и характера операций.

При вычислении выражений тип каждого операнда может быть преобразован к
другому типу. Преобразования типов могут быть неявными, при выполнении
операций и вызовах функций, или явными, при выполнении операций
приведения типов.

\subsection{Логические
выражения}\label{ux43bux43eux433ux438ux447ux435ux441ux43aux438ux435-ux432ux44bux440ux430ux436ux435ux43dux438ux44f}

Результатом логического выражения есть объект логического типа - истина
или ложь. Такие выражения соотносятся с выражениями из булевой алгебры и
используются в первую очередь в операторах ветвления и цикла.

\subsection{Текстовые
выражения}\label{ux442ux435ux43aux441ux442ux43eux432ux44bux435-ux432ux44bux440ux430ux436ux435ux43dux438ux44f}

Примечание автора

Текстовых выражений в С++ нет. По крайней мере, никакой информации на
этот счет найдено не было. Я не знаю, что сюда писать и нужно ли. Даже в
общем смысле, абстрагируясь от С++

\subsection{Математические
выражения}\label{ux43cux430ux442ux435ux43cux430ux442ux438ux447ux435ux441ux43aux438ux435-ux432ux44bux440ux430ux436ux435ux43dux438ux44f}

\textbf{Математическое выражение} - это совокупность знаков, описывающая
отношение между какими-то величинами. Результатом математических
выражений обычно является какое-либо число.
