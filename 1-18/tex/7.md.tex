\section{Знаки пунктуации, специальные символы и знаки операций в языке
С++}\label{ux437ux43dux430ux43aux438-ux43fux443ux43dux43aux442ux443ux430ux446ux438ux438-ux441ux43fux435ux446ux438ux430ux43bux44cux43dux44bux435-ux441ux438ux43cux432ux43eux43bux44b-ux438-ux437ux43dux430ux43aux438-ux43eux43fux435ux440ux430ux446ux438ux439-ux432-ux44fux437ux44bux43aux435-ux441}

\subsection{Знаки
пунктуации}\label{ux437ux43dux430ux43aux438-ux43fux443ux43dux43aux442ux443ux430ux446ux438ux438}

К \textbf{символам пунктуации} в языке \textbf{С++} относятся следующие:
\begin{verbatim}
! % ^ & * ( ) - + = { } | ~
[ ] \ ; ' : " < > ? , . / #
\end{verbatim}

Символы пунктуации в \textbf{C++} имеют синтаксическое и семантическое
значение для компилятора, однако сами по себе не указывают на операцию,
которая позволяет получить значение. Некоторые из них (по отдельности
или в сочетании) могут также быть операторами \textbf{C++} или иметь
значение для препроцессора.

\subsection{Управляющие
символы}\label{ux443ux43fux440ux430ux432ux43bux44fux44eux449ux438ux435-ux441ux438ux43cux432ux43eux43bux44b}

\textbf{Управляющие символы} (или как их ещё называют ---
\textbf{escape-последовательность}) --- символы которые выталкиваются в
поток вывода, с целью форматирования вывода или печати некоторых
управляющих знаков в С++.

\begin{longtable}[]{@{}
  >{\raggedright\arraybackslash}p{(\columnwidth - 2\tabcolsep) * \real{0.0776}}
  >{\raggedright\arraybackslash}p{(\columnwidth - 2\tabcolsep) * \real{0.9224}}@{}}
\toprule\noalign{}
\begin{minipage}[b]{\linewidth}\raggedright
символ
\end{minipage} & \begin{minipage}[b]{\linewidth}\raggedright
значение
\end{minipage} \\
\midrule\noalign{}
\endhead
\bottomrule\noalign{}
\endlastfoot
\texttt{\textbackslash{}a} & сигнал бипера (спикера) компьютера \\
\texttt{\textbackslash{}b} & возврат назад \\
\texttt{\textbackslash{}f} & следующая страница (англ. form feed) \\
\texttt{\textbackslash{}n} & новая строка \\
\texttt{\textbackslash{}r} & возврат каретки в начало строки \\
\texttt{\textbackslash{}t} & горизонтальная табуляция \\
\texttt{\textbackslash{}v} & вертикальная табуляция \\
\texttt{\textbackslash{}\textquotesingle{}} & одинарная кавычка \\
\texttt{\textbackslash{}"} & двойная кавычка \\
\texttt{\textbackslash{}\textbackslash{}} & обратная косая черта
(обратный слэш) \\
\texttt{\textbackslash{}?} & знак вопроса \\
\texttt{\textbackslash{}0} & нулевой символ \\
\texttt{\textbackslash{}ooo} & ASCII-символ в восьмеричной записи \\
\texttt{\textbackslash{}x\ hh} & ASCII-символ в шестнадцатеричной
записи \\
\texttt{\textbackslash{}x\ hhhh} & Символ Unicode в шестнадцатеричной
записи (только в для \emph{широких} (\texttt{wchar\_t}) и Unicode
(\texttt{char*\_t}) строках) \\
\end{longtable}
