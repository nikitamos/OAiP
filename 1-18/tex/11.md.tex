\section{Операнды в языках
программирования}\label{ux43eux43fux435ux440ux430ux43dux434ux44b-ux432-ux44fux437ux44bux43aux430ux445-ux43fux440ux43eux433ux440ux430ux43cux43cux438ux440ux43eux432ux430ux43dux438ux44f}

Комбинация \textbf{знаков операций} и \textbf{операндов}, результатом
которой является определенное значение, называется \textbf{выражением}.
Знаки операций определяют действия, которые должны быть выполнены над
операндами. Каждый операнд в выражении может быть выражением. Значение
выражения зависит от расположения знаков операций и круглых скобок в
выражении, а также от приоритета выполнения операций.

\textbf{Операнд} - любой объект, над которым проводится операция.

Например, операндом может быть - переменная или константа - литерал -
выражение вызова функции - выражение выбора элемента - любое другое
выражение, сформированное комбинацией операндов, знаков операций и
круглых скобок.

В языках высокого уровня зачастую за каждым операндом закреплен
определенный тип, будь то целочисленный, вещественный или какой-либо
другой.

Стоит отметить, что некоторые конструкции, даже не являясь выражениями в
привычном смысле (\emph{математические выражения}), тем не менее также
состоят из операндов:

\begin{enumerate}
\def\labelenumi{\arabic{enumi})}
\item
  Оператор ветвления - \texttt{if}.

\begin{Shaded}
\begin{Highlighting}[]
\ControlFlowTok{if} \OperatorTok{(}\NormalTok{ \_условие\_ }\OperatorTok{)} \OperatorTok{\{} \OperatorTok{...} \OperatorTok{\}}
\end{Highlighting}
\end{Shaded}

  \emph{условие} - операнд логического типа.
\item
  Оператор выбора - \texttt{switch}

\begin{Shaded}
\begin{Highlighting}[]
\ControlFlowTok{switch} \OperatorTok{(}\NormalTok{ \_значение\_ }\OperatorTok{)} \OperatorTok{\{}
  \ControlFlowTok{case}\NormalTok{ \_вариант}\DecValTok{1}\OperatorTok{\_:}
    \OperatorTok{...}
    \ControlFlowTok{break}\OperatorTok{;}
  \ControlFlowTok{case}\NormalTok{ \_вариант}\DecValTok{2}\OperatorTok{\_:}
    \OperatorTok{...}
    \ControlFlowTok{break}\OperatorTok{;}
  \OperatorTok{...}
  \ControlFlowTok{default}\OperatorTok{:}
    \OperatorTok{...}
    \ControlFlowTok{break}\OperatorTok{;}
\OperatorTok{\}}
\end{Highlighting}
\end{Shaded}

  \emph{значение} - операнд целого типа.
\item
  Оператор цикла с параметром - \texttt{for}.

\begin{Shaded}
\begin{Highlighting}[]
\ControlFlowTok{for} \OperatorTok{(}\NormalTok{\_инициализация\_}\OperatorTok{;}\NormalTok{ \_условие\_}\OperatorTok{;}\NormalTok{ \_модификация\_}\OperatorTok{)} \OperatorTok{\{} \OperatorTok{...} \OperatorTok{\}}
\end{Highlighting}
\end{Shaded}

  \emph{инициализация} - операнд, предназначенный для объявления
  переменных, используемых в цикле.

  \emph{условие} - операнд логического типа, определяет условие
  продолжения цикла.

  \emph{модификация} - операнд, выполняющийся после каждой итерации
  цикла.
\item
  Оператор циклов с пред- и постусловием - \texttt{while} и
  \texttt{do-while}.

\begin{Shaded}
\begin{Highlighting}[]
\ControlFlowTok{while} \OperatorTok{(}\NormalTok{ \_условие\_ }\OperatorTok{)} \OperatorTok{\{} \OperatorTok{...} \OperatorTok{\}}

\ControlFlowTok{do} \OperatorTok{\{} \OperatorTok{...} \OperatorTok{\}} \ControlFlowTok{while} \OperatorTok{(}\NormalTok{ \_условие\_ }\OperatorTok{);}
\end{Highlighting}
\end{Shaded}

  \emph{условие} - операнд логического типа.
\end{enumerate}
