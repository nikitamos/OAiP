\section{Операнды в языках программирования}

Комбинация \textbf{знаков операций} и \textbf{операндов}, результатом
которой является определенное значение, называется \textbf{выражением}.
Знаки операций определяют действия, которые должны быть выполнены над
операндами. Каждый операнд в выражении может быть выражением. Значение
выражения зависит от расположения знаков операций и круглых скобок в
выражении, а также от приоритета выполнения операций.

\textbf{Операнд} - любой объект, над которым проводится операция.

Например, операндом может быть - переменная или константа - литерал -
выражение вызова функции - выражение выбора элемента - любое другое
выражение, сформированное комбинацией операндов, знаков операций и
круглых скобок.

В языках высокого уровня зачастую за каждым операндом закреплен
определенный тип, будь то целочисленный, вещественный или какой-либо
другой.

Стоит отметить, что некоторые конструкции, даже не являясь выражениями в
привычном смысле (\emph{математические выражения}), тем не менее также
состоят из операндов:

\begin{enumerate}
\def\labelenumi{\arabic{enumi})}
\item
  Оператор ветвления - \texttt{if}.
\begin{minted}{C++}
  if ( _условие_ ) { ... }
\end{minted}

  \emph{условие} - операнд логического типа.
\item
  Оператор выбора - \texttt{switch}
\begin{minted}{C++}
  switch ( _значение_ ) {
      case _вариант1_:
        ...
        break;
      case _вариант2_:
        ...
        break;
      ...
      default:
        ...
        break;
    }
\end{minted}

  \emph{значение} - операнд целого типа.
\item
  Оператор цикла с параметром - \texttt{for}.
\begin{minted}{C++}
  for (_инициализация_; _условие_; _модификация_) { ... }
\end{minted}

  \emph{инициализация} - операнд, предназначенный для объявления
  переменных, используемых в цикле.

  \emph{условие} - операнд логического типа, определяет условие
  продолжения цикла.

  \emph{модификация} - операнд, выполняющийся после каждой итерации
  цикла.
\item
  Оператор циклов с пред- и постусловием - \texttt{while} и
  \texttt{do-while}.
\begin{minted}{C++}
  while ( _условие_ ) { ... }

    do { ... } while ( _условие_ );
\end{minted}

  \emph{условие} - операнд логического типа.
\end{enumerate}
