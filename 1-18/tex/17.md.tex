\section{Классификация бинарных
операций}\label{ux43aux43bux430ux441ux441ux438ux444ux438ux43aux430ux446ux438ux44f-ux431ux438ux43dux430ux440ux43dux44bux445-ux43eux43fux435ux440ux430ux446ux438ux439}

\textbf{Бинарная операция} - операция над двумя операндами и
возвращающая один результат.

К бинарным операциям в С++ относятся:

\begin{enumerate}
\def\labelenumi{\arabic{enumi})}
\tightlist
\item
  Присваивания
  (\texttt{=\ +=\ -=\ *=\ /=\ \%=\ \&=\ \textbar{}=\ \^{}=\ \textgreater{}\textgreater{}=\ \textless{}\textless{}=});
\item
  Арифметические, за исключением унарных \texttt{+}, \texttt{-}
  (\texttt{+\ -\ *\ /\ \%});
\item
  Побитовые, за исключением побитовой инверсии
  \texttt{\textasciitilde{}}
  (\texttt{\&\ \textbar{}\ \^{}\ \textless{}\textless{}\ \textgreater{}\textgreater{}});
\item
  Логические, за исключением унарного логического отрицания \texttt{!}
  (\texttt{\&\&\ \textbar{}\textbar{}});
\item
  Сравнения
  (\texttt{==\ !=\ \textless{}\ \textgreater{}\ \textless{}=\ \textgreater{}=\ \textless{}=\textgreater{}});
\item
  Обращения к члену ( \texttt{-\textgreater{}\ .});
\item
  Другие особые операторы:

  \begin{itemize}
  \tightlist
  \item
    Оператор разрешения области видимости (\texttt{::})
  \item
    Оператор \texttt{,} (запятая)
  \item
    Встроенный (неперегруженный) оператор \texttt{{[}{]}} (в С++23 этот
    оператор может иметь любую -арность, как и оператор вызова функции,
    но только в перегруженном виде. Встроенный оператор
    \texttt{E1{[}E2{]}} эквивалентен \texttt{*(E1\ +\ E2)})
  \end{itemize}
\end{enumerate}
