\section{Исходные и объектные модули, процессы компиляции и связывания (линковка)}

\textbf{Компиляция} --- сборка программы, включающая: - трансляцию всех
модулей программы, написанных на одном или нескольких исходных языках
программирования высокого уровня и/или языке ассемблера, в эквивалентные
программные модули на низкоуровневом языке, близком машинному коду
(абсолютный код, объектный модуль, иногда на язык ассемблера) или
непосредственно на машинном языке или ином двоичнокодовом низкоуровневом
командном языке; - последующую сборку исполняемой машинной программы, в
том числе вставка в программу кода всех функций, импортируемых из
статических библиотек и/или генерация кода запроса к ОС на загрузку
динамических библиотек, из которых программой функции будут вызываться.

Соответственно, программа, осуществляющая компиляцию, называется
\textbf{компилятором}.

Примеры компиляторов языка \textbf{С++}:
\begin{enumerate}
  \item \textbf{g++}, компилятор из
  набора инструментов (англ. toolchain) GCC;
  \item \textbf{clang++},
  компилятор из набора инструментов LLVM;
  \item \textbf{cl.exe} -
  программа-драйвер MSVC (Microsoft Visual C++).
\end{enumerate}
\textbf{Компоновщик} (редактор связей, линкер, сборщик) --- это
программа, которая производит компоновку («линковку», «сборку»):
принимает на вход один или несколько объектных модулей и собирает по ним
исполнимый модуль.

Примеры компоновщиков:
\begin{enumerate}
  \item \textbf{ld} - из набора инструментов GCC;
  \item \textbf{lld} - из набора инструментов LLVM;
  \item \textbf{link.exe} - из набора инструментов MSVC.
\end{enumerate}

\textbf{Исходный модуль} - программный модуль на исходном языке,
обрабатываемый транслятором.

\textbf{Объектный модуль} - двоичный файл, который содержит в себе
особым образом подготовленный исполняемый код, который может быть
объединён с другими объектными файлами при помощи редактора связей
(компоновщика) для получения готового исполняемого модуля, либо
библиотеки.

\textbf{Исполняемый модуль} (исполняемый файл) --- файл, который может
быть запущен на исполнение процессором под управлением операционной
системы.

\textbf{Препроцессор} --- программа для обработки текста. Может
существовать как отдельная программа, так и быть интегрированной в
компилятор. В любом случае, входные и выходные данные для препроцессора
имеют текстовый формат. Препроцессор преобразует текст в соответствии с
директивами препроцессора. Если текст не содержит директив
препроцессора, то текст остаётся без изменений.

В общем виде, сборка программы (С++) производится следующим образом:

\begin{enumerate}
\def\labelenumi{\arabic{enumi})}
\item
  Исходный модуль обрабатывается \textbf{препроцессором}.

  В этой фазе происходит текстовая обработка директив препроцессора
  (например, \texttt{\#include\ "foo/bar.h"} заменит строчку
  \texttt{\#include\ "foo/bar.h"} на содержимое файла по пути
  \texttt{./foo/bar.h});
\item
  \textbf{Фаза трансляции}. \textbf{Компилятор} на основе исходного
  модуля (файл с расширением \textbf{.cpp}) с внесенными изменениями
  создает \emph{объектный модуль}.
\item
  \textbf{Фаза компоновки}. Компоновщик собирает один или несколько
  \emph{объектных модулей}, файлы \emph{статических библиотек} и
  объединяет их в один исполняемый модуль.
\end{enumerate}

\subsection{Процесс компиляции}

Компиляция состоит из следующих этапов: - Лексический анализ -
объединение символов в лексемы; - Синтаксический анализ - построение
лексем в дерево разбора; - Семантический анализ - проверка и построение
семантической модели кода; - Оптимизация - перестроение программы для
увеличения ее быстродействия без видимых побочных эффектов; - Генерация
кода - создание итогового объектного модуля.

\subsection{Процесс компоновки (линковки)}

В объектном модуле сохраняется информация обо всех определенных функциях
и глобальных переменных. Эта информация сведена в таблицу символов
(англ. symbol table). При этом производятся необходимые искажения (англ.
mangling) имен функций для предотвращения коллизий имен (возникающих,
например, при перегрузке функций - в имя кодируются типы аргументов). По
таблицам символов компоновщик разрешает межмодульные зависимости, в
частности, подставляет реальные адреса функция в места их вызова.
Аналогично линкуются и глобальные переменные.
