\section{Идентификаторы в языке С++, правила наименования}

\subsection{Идентификаторы}

\textbf{Идентификатор} - это произвольно длинная последовательность
цифр, знаков нижнего подчеркивания букв латиницы верхнего и нижнего
регистров (и большинства символов Unicode, если присутствует поддержка
платформы), обозначающая имя какой-либо программной сущности (напр.
переменной, типа, метки и т. д.).

Пользовательские идентификаторы \textbf{не могут} начинаться с
\textbf{цифры} и содержать внутри себя \textbf{пробельные символы}.
Также пользовательский идентификатор не может совпадать с каким-либо
\textbf{ключевым словом} языка С++. Помимо этого не рекомендуется
создавать идентификаторы, начинающиеся с символа подчеркивания,
поскольку они могут являться внутренней деталью реализации стандартной
библиотеки С++ или определяемым компилятором макросом. Аналогично,
зарезервированными считаются идентификаторы с 2 нижними подчеркиваниями.

Несмотря на то, что стандарт \emph{не накладывает ограничений на длину}
идентификатора, она \emph{все равно может быть ограничена} компилятором
и/или компоновщиком. Например, \textbf{MSVC} и \textbf{Intel C++}
поддерживают идентификаторы длиной лишь в \textbf{2048} символов, в то
время как \textbf{GCC} никаких ограничений не накладывает.

\subsection{Правила названия}

Для увеличения читаемости кода, существуют различные соглашения о
названиях переменных типов и других программных сущностей. Вот,
например, правила, рекомендованные по стилю Google (google's codestyle):

\begin{small}
\begin{longtable}[]{@{}
  >{\raggedright\arraybackslash}p{(\columnwidth - 4\tabcolsep) * \real{0.1750}}
  >{\raggedright\arraybackslash}p{(\columnwidth - 4\tabcolsep) * \real{0.1583}}
  >{\raggedright\arraybackslash}p{(\columnwidth - 4\tabcolsep) * \real{0.6667}}@{}}
\toprule\noalign{}
\begin{minipage}[b]{\linewidth}\raggedright
Сущность
\end{minipage} & \begin{minipage}[b]{\linewidth}\raggedright
Стиль
\end{minipage} & \begin{minipage}[b]{\linewidth}\raggedright
Пример
\end{minipage} \\
\midrule\noalign{}
\endhead
\bottomrule\noalign{}
\endlastfoot
Локальная переменная & \textbf{snake\_case} &
\texttt{int\ bus\_queue\_length\ =\ 12;} \\
Тип (напр. структура) & \textbf{UpperCamelCase} &
\texttt{struct\ Unit\ \{\};} \\
Функция & \textbf{UpperCamelCase} &
\texttt{int\ GetMagicNumber()\ \{\ return\ 69;\ \}} \\
Параметр & \textbf{snake\_case} &
\texttt{int\ GetAnswerToTheUniverse(int\ years\_passed)\ \{\ return\ 42\ +\ 0\ *\ years\_passed;\ \}} \\
С макрос & \textbf{SCREAMING\_CASE} &
\texttt{\#define\ MY\_MACRO\_THAT\_SCARES\_SMALL\_CHILDREN\_AND\_ADULTS\_ALIKE} \\
Константа & k\textbf{UpperCamelCase} &
\texttt{const\ bool\ kIsAndroid\ =\ true;} \\
Поле класса & \textbf{snake\_case}\_ &
\texttt{class\ ProductData\ \{\ private:\ int\ number\_of\_customers\_;\ \};} \\
Член перечисления & k\textbf{UpperCamelCase} &
\texttt{enum\ Status\ \{\ kOk\ =\ 0,\ kBadAlloc,\ kRuntimeError\ \};} \\
Пространство имен & \textbf{snake\_case} &
\texttt{namespace\ foo\_bar\ \{\ ...\ \}} \\
\end{longtable}
\end{small}

Существуют и другие подходы, например \emph{венгерская нотация}, по
которой различные свойства объекта (например, типы переменных)
кодировались в названии этого объекта. Пример:

\begin{verbatim}
bool bBusy = false; // _b_oolean
const char *szOwner = "This is the owner"; // _z_ero-terminated _s_tring
\end{verbatim}

Также стоит придерживаться более общих правил:

\begin{enumerate}
\item
  Следует давать переменным и типам разборчивые имена, конкретные для
  предметной области.
\end{enumerate}

\begin{quote}
Например, создавая очередь покупателей не следует называть переменную
\texttt{queue}, вместо этого можно использовать более конкретное
\texttt{customer\_queue}.
\end{quote}

\begin{enumerate}
\item
  Следует избегать излишних аббревиатур.
\end{enumerate}

\begin{quote}
Например, вместо \texttt{class\ UQT\ \{\ ...\ \};} следует использовать
\texttt{UnitQuadTree\ \{\ ...\ \};}.
\end{quote}

\begin{enumerate}
\item
  Не стоит называть свои типы и переменные также, как называются типы в
  стандартной библиотеке.
\end{enumerate}

\begin{quote}
Это чревато коллизией имен и сложными в отладке ошибками.
\end{quote}
