\section{Унарные
операции}\label{ux443ux43dux430ux440ux43dux44bux435-ux43eux43fux435ux440ux430ux446ux438ux438}

\textbf{Унарная операция} - операция над одним операндом и возвращающая
один результат.

Рассмотрим унарные операции в \textbf{С++}.

\begin{enumerate}
\def\labelenumi{\arabic{enumi})}
\item
  Унарные `плюс' и `минус'

\begin{Verbatim}
  +a;
  -a;
\end{Verbatim}

  Унарный плюс не изменяет выражения, унарный минус меняет знак на
  противоположный.
\item
  Префиксный/постфиксный инкремент и декремент

\begin{Verbatim}
  ++a;
  a++;
  --a;
  a--;
\end{Verbatim}

  Прибавляет (инкремент) или отнимает (декремент) \(1\). Меняет операнд.
  В префиксной форме, сначала прибавляет/отнимает, потом возвращает
  значение измененной переменной. В постфиксной, сначала возвращает
  исходное значение переменной, потом прибавляет/отнимает.
\item
  Отрицание (логическое НЕ)

\begin{Verbatim}
  !a;
\end{Verbatim}

  Заменяет \texttt{true} на \texttt{false}, а \texttt{false} - на
  \texttt{true} и возвращает \textbf{логическое значение} - результат
  такой замены. Сам операнд не изменяется.
\item
  Побитовая инверсия

\begin{Verbatim}
  ~a;
\end{Verbatim}

  Заменяет \(0\) в двоичной записи числа на \(1\), а \(1\) - на \(0\) и
  возвращает \textbf{число} - результат такой замены. Сам операнд не
  изменяется.
\item
  Взятие адреса и косвенная адресация

\begin{Verbatim}
  int *address = &a;
  *a = 42;
\end{Verbatim}

  Взятие адреса применяется к переменной и возвращает адрес первой
  ячейки памяти, где хранится эта переменная, в виде указателя.
  Косвенная адресация позволяет прочитать и записать по адресу,
  хранящемуся в указателе.
\item
  Оператор \texttt{sizeof()} - получение размера типа

\begin{Verbatim}
  sizeof(a)
\end{Verbatim}
  Возвращает размер типа(типа выражения) в байтах. Стандарт допускает,
  что байт может не быть равен 8 бит, поэтому точнее говорить, что
  \texttt{sizeof} возвращает размер типа в адресуемых ячейках памяти.
\end{enumerate}

\begin{enumerate}
\def\labelenumi{\arabic{enumi})}
\setcounter{enumi}{6}
\item
  Оператор приведения типов в стиле С
  \begin{Verbatim}
    int a = 42;
    double c_style = (double)a;
  \end{Verbatim}
\item
  выделение и освобождение памяти - \texttt{new} и \texttt{delete}

\begin{Verbatim}
  int *heap_allocated = new int;
    *heap_allocated = 42;

    //...

    delete heap_allocated;
\end{Verbatim}
\end{enumerate}

\ldots и хватит\ldots{}
