\section{Унарные
операции}\label{ux443ux43dux430ux440ux43dux44bux435-ux43eux43fux435ux440ux430ux446ux438ux438}

\textbf{Унарная операция} - операция над одним операндом и возвращающая
один результат.

Рассмотрим унарные операции в \textbf{С++}.

\begin{enumerate}
\def\labelenumi{\arabic{enumi})}
\item
  Унарные `плюс' и `минус'

\begin{Shaded}
\begin{Highlighting}[]
\OperatorTok{+}\NormalTok{a}\OperatorTok{;}
\OperatorTok{{-}}\NormalTok{a}\OperatorTok{;}
\end{Highlighting}
\end{Shaded}

  Унарный плюс не изменяет выражения, унарный минус меняет знак на
  противоположный.
\item
  Префиксный/постфиксный инкремент и декремент

\begin{Shaded}
\begin{Highlighting}[]
\OperatorTok{++}\NormalTok{a}\OperatorTok{;}
\NormalTok{a}\OperatorTok{++;}
\OperatorTok{{-}{-}}\NormalTok{a}\OperatorTok{;}
\NormalTok{a}\OperatorTok{{-}{-};}
\end{Highlighting}
\end{Shaded}

  Прибавляет (инкремент) или отнимает (декремент) \(1\). Меняет операнд.
  В префиксной форме, сначала прибавляет/отнимает, потом возвращает
  значение измененной переменной. В постфиксной, сначала возвращает
  исходное значение переменной, потом прибавляет/отнимает.
\item
  Отрицание (логическое НЕ)

\begin{Shaded}
\begin{Highlighting}[]
\OperatorTok{!}\NormalTok{a}\OperatorTok{;}
\end{Highlighting}
\end{Shaded}

  Заменяет \texttt{true} на \texttt{false}, а \texttt{false} - на
  \texttt{true} и возвращает \textbf{логическое значение} - результат
  такой замены. Сам операнд не изменяется.
\item
  Побитовая инверсия

\begin{Shaded}
\begin{Highlighting}[]
\OperatorTok{\textasciitilde{}}\NormalTok{a}\OperatorTok{;}
\end{Highlighting}
\end{Shaded}

  Заменяет \(0\) в двоичной записи числа на \(1\), а \(1\) - на \(0\) и
  возвращает \textbf{число} - результат такой замены. Сам операнд не
  изменяется.
\item
  Взятие адреса и косвенная адресация

\begin{Shaded}
\begin{Highlighting}[]
\DataTypeTok{int} \OperatorTok{*}\NormalTok{address }\OperatorTok{=} \OperatorTok{\&}\NormalTok{a}\OperatorTok{;}
\OperatorTok{*}\NormalTok{a }\OperatorTok{=} \DecValTok{42}\OperatorTok{;}
\end{Highlighting}
\end{Shaded}

  Взятие адреса применяется к переменной и возвращает адрес первой
  ячейки памяти, где хранится эта переменная, в виде указателя.
  Косвенная адресация позволяет прочитать и записать по адресу,
  хранящемуся в указателе.
\item
  Оператор \texttt{sizeof()} - получение размера типа

\begin{Shaded}
\begin{Highlighting}[]
\KeywordTok{sizeof}\OperatorTok{(}\NormalTok{a}\OperatorTok{)}
\end{Highlighting}
\end{Shaded}

  Возвращает размер типа(типа выражения) в байтах. Стандарт допускает,
  что байт может не быть равен 8 бит, поэтому точнее говорить, что
  \texttt{sizeof} возвращает размер типа в адресуемых ячейках памяти.
\end{enumerate}

\begin{enumerate}
\def\labelenumi{\arabic{enumi})}
\setcounter{enumi}{6}
\item
  Оператор приведения типов в стиле С
  \texttt{cpp\ \ \ \ \ int\ a\ =\ 42;\ \ \ \ \ double\ c\_style\ =\ (double)a;}
\item
  выделение и освобождение памяти - \texttt{new} и \texttt{delete}

\begin{Shaded}
\begin{Highlighting}[]
\DataTypeTok{int} \OperatorTok{*}\NormalTok{heap\_allocated }\OperatorTok{=} \KeywordTok{new} \DataTypeTok{int}\OperatorTok{;}
\OperatorTok{*}\NormalTok{heap\_allocated }\OperatorTok{=} \DecValTok{42}\OperatorTok{;}

\CommentTok{//...}

\KeywordTok{delete}\NormalTok{ heap\_allocated}\OperatorTok{;}
\end{Highlighting}
\end{Shaded}
\end{enumerate}

\ldots и хватит\ldots{}
