\section{Виды констант в языке С++}

Под \textbf{константой} понимается какая-либо \textbf{неизменяемая}
величина.

Можно выделить следующие виды констант:
\begin{enumerate}
  \item Макросы;
  \item Литералы;
  \item Собственно константы (с ключевым словом const);
  \item Выражения constexpr;
  \item Константы, члены перечисления.
\end{enumerate}

\subsection{Литералы}

Литералы - непосредственные значения определенного типа, встроенные в
программный код.

Литералы могут иметь различный тип:
\begin{enumerate}
  \item целый;
  \item вещественный;
  \item логический;
  \item символьный;
  \item типа указателя (\minline{nullptr});
  \item строковый;
  \item (\emph{необязательно}) пользовательского типа (при
    соответствующем переопределенном операторе в пользовательском типе).
\end{enumerate}

\begin{tiny}
  \begin{longtable}[]{@{}
    >{\raggedright\arraybackslash}p{(\columnwidth - 4\tabcolsep) * \real{0.1341}}
    >{\raggedright\arraybackslash}p{(\columnwidth - 4\tabcolsep) * \real{0.7095}}
    >{\raggedright\arraybackslash}p{(\columnwidth - 4\tabcolsep) * \real{0.2365}}@{}}
    \toprule\noalign{}
    \begin{minipage}[b]{\linewidth}\raggedright
      Тип литерала
    \end{minipage} & \begin{minipage}[b]{\linewidth}\raggedright
      Формат
    \end{minipage} & \begin{minipage}[b]{\linewidth}\raggedright
      Пример
    \end{minipage} \\
    \midrule\noalign{}
    \endhead
    \bottomrule\noalign{}
    \endlastfoot
    целый & префикс \texttt{0} - 8-ричноепрефикс\par \texttt{0x} -
    16-ричноепрефикс\par \texttt{0b} - 2-чное\par суффикс \texttt{l} - не меньший,
    чем \texttt{long\ int}\par суффикс \texttt{u} - не меньший, чем
    \texttt{unsigned\ int}\par суффикс \texttt{ll} - не меньший чем
    \texttt{long\ long\ int}\par одновременно \texttt{ll} и \texttt{u} - не
    меньший, чем \texttt{unsigned\ long\ long\ int}\par Регистр символов неважен;
    группы цифр можно разделять \texttt{\textquotesingle{}} &
    \texttt{42}\par \texttt{052}\par \texttt{0x2a}\par \texttt{0x2A}\par \texttt{0b101010} (с
    \textbf{С++14})\par \texttt{18446744073709550592ull}\par \texttt{1\textquotesingle{}000\textquotesingle{}000\textquotesingle{}000ULL} \\
    вещественный & позволительна запись в экспоненциальном
    (\(m*10^{\pm t}\)) виде: \texttt{mE±t}, (если \(t\ge 0\), \(+\) можно
    опустить: \texttt{mEt})\par суффикс \texttt{f} - тип \texttt{float}\par суффикс
    \texttt{l} - тип \texttt{long\ double}\par иначе - \texttt{double}\par регистр
    символов не имеет значения; группы цифр можно разделять
    \texttt{\textquotesingle{}} &
    \texttt{1.0056}\par \texttt{3.1415E12}\par \texttt{0.0f}\par \texttt{1\textquotesingle{}000\textquotesingle{}000.123\textquotesingle{}812L} \\
    логический & \texttt{true} - \emph{истина}\par \texttt{false} - \emph{ложь} &
    \texttt{true} \\
    литерал-указатель & \texttt{nullptr} - предпочтительная замена
    \texttt{0} и макросу \texttt{NULL} & \texttt{nullptr} \\
    символьный & без префикса - \texttt{char}\par \texttt{u8} -
    \texttt{char8\_t}(c \textbf{C++17})\par \texttt{L} -
    \texttt{wchar\_t}\par \texttt{u} - \texttt{char16\_t}\par \texttt{U} -
    \texttt{char32\_t}\par типы \texttt{char*\_t} хранят символ Unicode в
    кодировке UTF-* \par разрешены escape-последовательности &
    \texttt{\textquotesingle{}a\textquotesingle{}}\par \texttt{\textquotesingle{}\textbackslash{}n\textquotesingle{}}\par \texttt{u\textquotesingle{}=\textquotesingle{}}\par \texttt{u8\textquotesingle{}W\textquotesingle{}}\par \texttt{L\textquotesingle{}ъ\textquotesingle{}}\par \texttt{U\textquotesingle{}*\textquotesingle{}}\par \\
    строковый & префиксы аналогичны символьным литералам, но обозначают тип
    символов строки\par суффикс \texttt{s} используется для создания литерала
    объекта \texttt{std::*string} (\texttt{std::string},
    \texttt{std::wstring}, \texttt{std::u8string}, и т.д.)\par также существуют
    \emph{сырые строки}, \par Они задаются префиксом \texttt{R} и ограничиваются \par
    \texttt{R"символы(....)символы"} &
    \texttt{"hello"}\par \texttt{u8"Hello,\ I\textquotesingle{}m\ UTF-8"}\par \texttt{L"Long\ string\ literal"}\par \texttt{R"(Prepare\ thyself!)"}\par \texttt{R"lit(A\ "(string)")lit"} \\
  \end{longtable}
\end{tiny}
  
\subsection{Именованные константы}
  
Символические константы - именованные константы, используются чтобы не
использовать \emph{магические числа} т.к. оные не несут смысла без
контекста.

\begin{enumerate}
\item
  Макросы - подставляются препроцессором. Не рекомендуются к
  использованию.
\begin{minted}{C++}
  #define ANSWER_TO_THE_UNIVERSE 42
\end{minted}
\item
  \texttt{const} - основной способ показать неизменяемость в С++.

  \begin{itemize}
  \item
    Константные переменные должны быть инициализированы, когда вы их
    определяете, после этого это значение не может быть изменено с
    помощью присваивания;
  \item
    Объявление переменной как const предотвращает непреднамеренное
    изменение ее значения;
  \item
    Константные переменные могут быть инициализированы из других
    переменных (включая неконстантные).
  \end{itemize}

  \texttt{const} часто используется с параметрами функции, что
  гарантирует, что функция не изменит значение аргумента. Впрочем,
  данное свойство полезно только при передаче \emph{по ссылке} или
  \emph{по указателю}: при передаче по значению все аргументы функции
  копируются и функция в любом случае не может изменить значение извне:
\begin{minted}{C++}
    void ByValue(int i) { i = 42; }
    void ByReference(int &i) { i = 69; }
    void ByPointer(int *i) { *i = 34; }

    int main() {
      int a = 0;
      int b = 0;
      int c = 0;

      ByValue(a);
      ByReference(b);
      ByPointer(&c);

      // a == 0, b == 69, c == 34

      return 0;
    }
\end{minted}
\item
  Ключевое слово \texttt{constexpr} появилось с \textbf{С++11} и
  позволяет показать \emph{константу времени компиляции}, т.е. значения,
  помеченные \texttt{constexpr} компилятор вправе вычислить во время
  компиляции. При этом было бы ошибкой попытаться инициализировать
  \texttt{constexpr} константу не \texttt{constexpr} выражением.
\item
  Члены перечисления также являются символическими константами связанным
  с ними числовым значениям:
\begin{minted}{C++}
  enum MyEnum {
      kZero = 0,
      kOne,
      kTwo,
      kFive = 5,
      kSix
    };
    // ...
    std::cout << kFive; // 5
\end{minted}
\end{enumerate}
