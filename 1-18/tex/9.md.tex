\section{Виды констант в языке
С++}\label{ux432ux438ux434ux44b-ux43aux43eux43dux441ux442ux430ux43dux442-ux432-ux44fux437ux44bux43aux435-ux441}

Под \textbf{константой} понимается какая-либо \textbf{неизменяемая}
величина.

Можно выделить следующие виды констант: - Макросы; - Литералы; -
Собственно константы (с ключевым словом const); - Выражения constexpr; -
Константы, члены перечисления.

\subsection{Литералы}\label{ux43bux438ux442ux435ux440ux430ux43bux44b}

Литералы - непосредственные значения определенного типа, встроенные в
программный код.

Литералы могут иметь различный тип: - целый; - вещественный; -
логический; - типа указателя (\texttt{nullptr}); - символьный; -
строковый; - (\emph{необязательно}) пользовательского типа (при
соответствующем переопределенном операторе в пользовательском типе).

\begin{longtable}[]{@{}
  >{\raggedright\arraybackslash}p{(\columnwidth - 4\tabcolsep) * \real{0.0341}}
  >{\raggedright\arraybackslash}p{(\columnwidth - 4\tabcolsep) * \real{0.7295}}
  >{\raggedright\arraybackslash}p{(\columnwidth - 4\tabcolsep) * \real{0.2365}}@{}}
\toprule\noalign{}
\begin{minipage}[b]{\linewidth}\raggedright
Тип литерала
\end{minipage} & \begin{minipage}[b]{\linewidth}\raggedright
Формат
\end{minipage} & \begin{minipage}[b]{\linewidth}\raggedright
Пример
\end{minipage} \\
\midrule\noalign{}
\endhead
\bottomrule\noalign{}
\endlastfoot
целый & префикс \texttt{0} - 8-ричноепрефикс \texttt{0x} -
16-ричноепрефикс \texttt{0b} - 2-чноесуффикс \texttt{l} - не меньший,
чем \texttt{long\ int}суффикс \texttt{u} - не меньший, чем
\texttt{unsigned\ int}суффикс \texttt{ll} - не меньший чем
\texttt{long\ long\ int}одновременно \texttt{ll} и \texttt{u} - не
меньший, чем \texttt{unsigned\ long\ long\ int}Регистр символов неважен;
группы цифр можно разделять \texttt{\textquotesingle{}} &
\texttt{42}\texttt{052}\texttt{0x2a}\texttt{0x2A}\texttt{0b101010} (с
\textbf{С++14})\texttt{18446744073709550592ull}\texttt{1\textquotesingle{}000\textquotesingle{}000\textquotesingle{}000ULL} \\
вещественный & позволительна запись в экспоненциальном
(\(m*10^{\pm t}\)) виде: \texttt{mE±t}, (если \(t\ge 0\), \(+\) можно
опустить: \texttt{mEt})суффикс \texttt{f} - тип \texttt{float}суффикс
\texttt{l} - тип \texttt{long\ double}иначе - \texttt{double}регистр
символов не имеет значения; группы цифр можно разделять
\texttt{\textquotesingle{}} &
\texttt{1.0056}\texttt{3.1415E12}\texttt{0.0f}\texttt{1\textquotesingle{}000\textquotesingle{}000.123\textquotesingle{}812L} \\
логический & \texttt{true} - \emph{истина}\texttt{false} - \emph{ложь} &
\texttt{true} \\
литерал-указатель & \texttt{nullptr} - предпочтительная замена
\texttt{0} и макросу \texttt{NULL} & \texttt{nullptr} \\
символьный & без префикса - \texttt{char}\texttt{u8} -
\texttt{char8\_t}(c \textbf{C++17})\texttt{L} -
\texttt{wchar\_t}\texttt{u} - \texttt{char16\_t}\texttt{U} -
\texttt{char32\_t}типы \texttt{char*\_t} хранят символ Unicode в
кодировка UTF-\emph{разрешены }escape-последовательности* &
\texttt{\textquotesingle{}a\textquotesingle{}}\texttt{\textquotesingle{}\textbackslash{}n\textquotesingle{}}\texttt{u\textquotesingle{}猫\textquotesingle{}}\texttt{u8\textquotesingle{}W\textquotesingle{}}\texttt{L\textquotesingle{}β\textquotesingle{}}\texttt{U\textquotesingle{}🍌\textquotesingle{}} \\
строковый & префиксы аналогичны символьным литералам, но обозначают тип
символов строкисуффикс \texttt{s} используется для создания литерала
объекта \texttt{std::*string} (\texttt{std::string},
\texttt{std::wstring}, \texttt{std::u8string}, и т.д.)также существуют
\emph{сырые строки}, Они задаются префиксом \texttt{R} и ограничиваются
\texttt{R"символы(....)символы"} &
\texttt{"hello"}\texttt{u8"Hello,\ I\textquotesingle{}m\ UTF-8"}\texttt{L"Long\ string\ literal"}\texttt{R"(Prepare\ thyself!)"}\texttt{R"lit(A\ "(string)")lit"} \\
\end{longtable}

\subsection{Именованные
константы}\label{ux438ux43cux435ux43dux43eux432ux430ux43dux43dux44bux435-ux43aux43eux43dux441ux442ux430ux43dux442ux44b}

Символические константы - именованные константы, используются чтобы не
использовать \emph{магические числа} т.к. оные не несут смысла без
контекста.

\begin{enumerate}
\def\labelenumi{\arabic{enumi})}
\item
  Макросы - подставляются препроцессором. Не рекомендуются к
  использованию.

\begin{Shaded}
\begin{Highlighting}[]
\PreprocessorTok{\#define ANSWER\_TO\_THE\_UNIVERSE }\DecValTok{42}
\end{Highlighting}
\end{Shaded}
\item
  \texttt{const} - основной способ показать неизменяемость в С++.

  \begin{itemize}
  \tightlist
  \item
    Константные переменные должны быть инициализированы, когда вы их
    определяете, после этого это значение не может быть изменено с
    помощью присваивания;
  \item
    Объявление переменной как const предотвращает непреднамеренное
    изменение ее значения;
  \item
    Константные переменные могут быть инициализированы из других
    переменных (включая неконстантные).
  \end{itemize}

  \texttt{const} часто используется с параметрами функции, что
  гарантирует, что функция не изменит значение аргумента. Впрочем,
  данное свойство полезно только при передаче \emph{по ссылке} или
  \emph{по указателю}: при передаче по значению все аргументы функции
  копируются и функция в любом случае не может изменить значение извне:
  ```cpp void ByValue(int i) \{ i = 42; \} void ByReference(int \&i) \{
  i = 69; \} void ByPointer(int \emph{i) \{ }i = 34; \}

  int main() \{ int a = 0; int b = 0; int c = 0;

  ByValue(a); ByReference(b); ByPointer(\&c);

  // a == 0, b == 69, c == 34

  return 0; \} ```
\item
  Ключевое слово \texttt{constexpr} появилось с \textbf{С++11} и
  позволяет показать \emph{константу времени компиляции}, т.е. значения,
  помеченные \texttt{constexpr} компилятор вправе вычислить во время
  компиляции. При этом было бы ошибкой попытаться инициализировать
  \texttt{constexpr} константу не \texttt{constexpr} выражением.
\item
  Члены перечисления также являются символическими константами связанным
  с ними числовым значениям:

\begin{Shaded}
\begin{Highlighting}[]
\KeywordTok{enum}\NormalTok{ MyEnum }\OperatorTok{\{}
\NormalTok{  kZero }\OperatorTok{=} \DecValTok{0}\OperatorTok{,}
\NormalTok{  kOne}\OperatorTok{,}
\NormalTok{  kTwo}\OperatorTok{,}
\NormalTok{  kFive }\OperatorTok{=} \DecValTok{5}\OperatorTok{,}
\NormalTok{  kSix}
\OperatorTok{\};}
\CommentTok{// ...}
\BuiltInTok{std::}\NormalTok{cout }\OperatorTok{\textless{}\textless{}}\NormalTok{ kFive}\OperatorTok{;} \CommentTok{// 5}
\end{Highlighting}
\end{Shaded}
\end{enumerate}
