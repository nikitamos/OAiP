\section{Алфавит языка С++. Лексемы}

\subsection{Алфавит}

Алфавит языка C++ для файлов исходного кода обязательно (по стандарту
\textbf{С++23}) включает в себя:

\begin{enumerate}
\item
  Строчную базовую латиницу: \texttt{a-z};
\item
  Прописную базовую латиницу: \texttt{A-Z};
\item
  Арабские цифры: \texttt{0-9};
\item
  Специальные знаки:
  \texttt{,.;:?!\textquotesingle{}"\textbar{}/\textbackslash{}\textasciitilde{}\_\^{}()\{\}{[}{]}\textless{}\textgreater{}\#\%\&-=+*};
\item
  Пробельные и управляющие символы (приведены также их
  \emph{escape-последовательности - символы которые выталкиваются в
  поток вывода, с целью форматирования вывода или печати некоторых
  управляющих знаков С++}):

  \begin{itemize}
  \item
    \texttt{\textbackslash{}t} - табуляция;
  \item
    \texttt{\textbackslash{}v} - вертикальная табуляция;
  \item
    \texttt{\textbackslash{}f} - смена страницы (англ. \emph{form
    feed});
  \item
    \texttt{\textbackslash{}n} - перевод строки;
  \item
    \texttt{\textbackslash{}0} - null-символ;
  \item
    \texttt{\textbackslash{}b} - возврат на шаг;
  \item
    \texttt{\textbackslash{}a} - звуковой сигнал (англ. \emph{bell});
  \item
    \texttt{\textbackslash{}r} - перевод каретки;
  \item
    \texttt{} - пробел.
  \end{itemize}
\end{enumerate}

Впрочем, компиляторы могут поддерживать и более расширенный алфавит,
например GCC поддерживает также символы UTF-8, например кириллицу.
Строки и комментарии могут состоять вообще говоря из любых символов,
поддерживаемых платформой.

\subsection{Лексемы (a.k.a. Tokens)}

Лексема (иначе \emph{токен}, от англ. token) - минимальный лексический
элемент языка С++ на этапе компиляции.

Категории лексем:
\begin{enumerate}
  \item Идентификаторы;
  \item Ключевые слова;
  \item Литерал;
  \item Операторы;
  \item Знаки пунктуации (\texttt{;\ ,\ \{\}\ ()} и т.д.).
\end{enumerate}

\textbf{Идентификатор} - это произвольно длинная последовательность
цифр, знаков нижнего подчеркивания букв латиницы верхнего и нижнего
регистров (и большинства символов Unicode, если присутствует поддержка
платформы), обозначающая имя какой-либо программной сущности (напр.
переменной, типа, метки и т. д.).

\textbf{Ключевое слово} - это предварительно определенный
зарезервированный идентификатор, имеющий специальные значение. Его
нельзя использовать в качестве идентификатора в программе.

\textbf{Литерал} - это непосредственное значение (целочисленное,
вещественное, символьное, логическое, литерал-указатель
\texttt{nullptr}, строковое).

\textbf{Оператор} - элемент программы, который контролирует способ и
порядок обработки объектов.

\textbf{Знаки пунктуации} сами по себе смысла не несут, однако они
являются составными частями операторов, и иных синтаксических
конструкций.

\subsubsection{Идентификаторы}

\textbf{Идентификатор} - это произвольно длинная последовательность
цифр, знаков нижнего подчеркивания букв латиницы верхнего и нижнего
регистров (и большинства символов Unicode, если присутствует поддержка
платформы), обозначающая имя какой-либо программной сущности (напр.
переменной, типа, метки и т. д.).

Пользовательские идентификаторы не могут начинаться с цифры и содержать
внутри себя пробельные символы. Также пользовательский идентификатор не
может совпадать с каким-либо ключевым словом языка С++. Помимо этого не
рекомендуется создавать идентификаторы, начинающиеся с символа
подчеркивания, поскольку они могут являться внутренней деталью
реализации стандартной библиотеки С++ или определяемым компилятором
макросом.

\subsubsection{Ключевые слова}

\textbf{Ключевое слово} - это предварительно определенный
зарезервированный идентификатор, имеющий специальные значение. Его
нельзя использовать в качестве идентификатора в программе.
