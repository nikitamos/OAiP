\section{Типы данных – простые и составные. Агрегирование данных}
\section{Указатели и ссылки в языке С++. Семантика копирования и перемещения}
\section{Функции в языке С++. Понятие метода. Рекурсия}
\section{Указатели на функции в языке С++}
\section{Библиотечные модули пользователя. Назначение модуля. Структура модуля. Синтаксис и назначение разделов модуля. Пример}
\section{Программирование статистических и динамических библиотек.  Подключение в различных средах}
\section{Статические и динамические структуры данных. Последовательности и динамические массивы. Реализация операций над последовательностями}
\section{Линейные списки – основные операции. Реализация списков на основе динамических структур}
\section{Двусвязный список и его программная реализация. Кольцевые списки.  Упорядоченные списки и перестройка списков}
