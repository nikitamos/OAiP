\section{Множества. Основные операции: проверка, добавление, удаление. Реализация списком, массивом, битовой картой}
\section{Словари и хеш-таблицы. Хеширование, хеш-функции. Коллизии хеширования}
\textbf{Хеширование}~--- процесс преобразования входных данных произвольной длины (ключа) в фиксированный размер выходных данных (хеш-код) с помощью
\textit{хеш-функции}.

\textbf{Хеш-функция}~--- функция, преобразующая массив входных данных произвольного размера в выходную строку фиксированной длины, называемую хешем.

Заметим, что так как под хешем понимается (битовая) строка фиксированной длины, то, вообще говоря, по принципу Дирихле существуют различные данные,
имеющие одинаковый хеш. Такое совпадение хешей часто называют <<коллизией хеширования>> и все алгоритмы, опирающиеся на хеш-функции, должны это
учитывать. Несмотря на это, хеш-функции широко используются в различных ветвях информатики. Рассмотрим одно из их приложений.

\subsection{Словари. Хеш-таблицы}
\textbf{Ассоциативным массивом} (словарем\footnote{Иногда конкретизируют, что словарь~--- это ассоциативный массив именно со \textit{строками} в качестве ключа.
Но как тогда объяснить, что ассоциативный массив в том же C\# называется \href{https://learn.microsoft.com/en-us/dotnet/api/system.collections.generic.dictionary-2}{Dictionary<TKey, TValue>}?})
называется абстрактный тип данных, хранящий пары <<ключ-значение>> (ключ не может повторяться) и поддерживающий следующие операции:
\begin{enumerate}
  \item \mverb{insert(key, value)}~--- добавляет пару в коллекцию;
  \item \mverb{find(key)}~--- извлекает значение по его ключу;
  \item \verb|delete(key)|~--- удаляет пару по ключу.
\end{enumerate}

С математической точки зрения ассоциативный массив задает конечное отображение из подмножества всех объектов типа ключей в подмножество всех
объектов типа значений.

Разумеется, практический интерес зачастую представляют собой только такие реализации словарей, в которых базовые операции выполняются эффективно~---
по крайней мере логарифмически (в среднем). Такие асимптотические оценки можно получить, используя сбалансированные деревья поиска или,
(о чем речь пойдет далее) хеш-таблицы.

\textbf{Хеш-таблица}~--- это ассоциативный массив, в котором местоположение элемента массива зависит от значения самого элемента.
Связь между значением элемента и его позицией в хеш-таблице задает \textit{хеш-функция}.
Так как при хорошо подобранной хеш-функции вероятность коллизий достаточно мала, хеш-таблицы могут предоставлять функционал ассоциативного массива
со средней временной сложностью $O(1)$, что много лучше $O(\log{n})$ у сбалансированных деревьев поиска. Недостаток хеш-таблиц заключается в
деградации их временной сложности до $O(n)$ с увеличением числа коллизий, вне зависимости от способа их разрешения.

Выбор хеш-функции напрямую влияет на эффективность хеш-таблицы. Хорошая хеш-функция должна распределять значения хешей для всевозможных входных данных равномерно,
быть простой в вычислении и, в идеале, обладать лавинным эффектом. Впрочем помимо этого для минимизации коллизий следует держать хеш-таблицу лишь
частично заполненной ($\alpha = \frac{\text{число ключей в таблице}}{\text{размер таблицы}}, \alpha \leq 0.7$).

Коллизии в хеш-таблице можно разрешать различными способами, и это серьезно влияет на внутреннюю архитектуру структуры данных.

\section{Обработка (разрешение) коллизий хеширования: прямое связывание, "открытая адресация"}
\label{sec:hashtable-collisions}
\section{Иерархические списки, деревья. Основные определения, связанные с деревьями}
\section{Бинарные деревья - основные понятия. Основные операции с бинарными деревьями}
\section{Алгоритмы обхода дерева (поиск в неупорядоченном дереве)}
\section{Добавление и удаление элементов дерева}
