\documentclass[14pt, a4paper]{extarticle}
\usepackage[russian]{babel}
\usepackage[T2A]{fontenc}
\usepackage{indentfirst}
\usepackage{hyperref}
\usepackage{makecell}
\usepackage{multirow}
\usepackage{amsthm}
\usepackage{amsmath}

\begin{document}
\title{Ответы по матлогике}
\maketitle

\newcommand{\itembf}[1]{\item \textbf{#1}}
\newtheorem*{defin}{Определение}
\newtheorem*{theorem}{Теорема}

\newpage
\tableofcontents
\newpage

\section{Булевы функции}
\subsection{Булевы функции. Способы задания, основные свойства.}

\subparagraph{Булева} (или логическая) \textbf{функция} $y=f(x_1, x_2, ..., x_n)$ от $n$  переменных $x_1, x_2, ..., x_n$ --- такая функция, что и аргументы и функция могут принимать только значения $0$ (истина) либо $1$ (ложь).

\subparagraph{Способы задания}
\begin{enumerate}
  \item {Таблицей истинности}
  \item {Вектором значений}
  \item {Номером}
  \item {Формулами}
\end{enumerate}
\subparagraph{Приоритет выполнения операций}

\subsection{Булевы функции одной и двух переменных}

\subparagraph{Одной переменной}


\begin{center}
\begin{tabular}{|c|c|c|c|}
  \hline
  Функция & $x=0$ & $x=1$ & Словесное описание \\
  \hline
  $f_0 \equiv 0$ & 0 & 0 & \makecell{Константный ноль\\(противоречие)} \\
  \hline
  $f_1 = x$ & 0 & 1 & \makecell{Повторение аргумента \\ (тождественная функция)} \\
  \hline
  $f_2 = \overline x$ & 1 & 0 & \makecell{Инверсия (отрицание)\\ аргумента} \\
  \hline
  $f_3 \equiv 1$ & 1 & 1 & \makecell{Константная единица\\(тавтология)} \\
  \hline
\end{tabular}
\end{center}

\subparagraph{Двух переменных}
Обратите внимание на порядок следования переменных в таблице!

\begin{center}
\begin{tabular}{|c|c|c|c|c|c|}
\hline
$x_1$ & 0 & 0 & 1 & 1 & \multirow{2}{*}{Описание} \\
\cline{1-5}
$x_2$ & 0 & 1 & 0 & 1 & \\
\hline
\hline
$f_0$ & 0 & 0 & 0 & 0 & \makecell{Противоречие} \\
\hline
$f_1$ & 0 & 0 & 0 & 1 & \makecell{Конъюнкция \\ $f = x_1 \land x_2$} \\
\hline
$f_2$ & 0 & 0 & 1 & 0 & \makecell{$f = \overline{x_1 \rightarrow x_2}$} \\
\hline
$f_3$ & 0 & 0 & 1 & 1 & \makecell{$f = x_1$} \\
\hline
$f_4$ & 0 & 1 & 0 & 0 & \makecell{$f = \overline{x_2 \rightarrow x_1}$} \\
\hline
$f_5$ & 0 & 1 & 0 & 1 & \makecell{$f = x_2$} \\
\hline
$f_6$ & 0 & 1 & 1 & 0 & \makecell{XOR (исключающее или)\\ $f = x_1 \oplus x_2$} \\
\hline
$f_7$ & 0 & 1 & 1 & 1 & \makecell{Дизъюнкция \\ $f = x_1 \lor x_2$} \\
\hline
$f_8$ & 1 & 0 & 0 & 0 & \makecell{Стрелка Пирса (отр. дизъюнкции) \\
$f = x_1 \downarrow x_2 = \overline{x_1 \lor x_2}$} \\
\hline
$f_9$ & 1 & 0 & 0 & 1 & \makecell{Эквиваленция \\ $ f = (x_1 \leftrightarrow x_2) = (x_1 \rightarrow x_2) \land (x_2 \rightarrow x_1)$} \\
\hline
$f_{10}$ & 1 & 0 & 1 & 0 & \makecell{$f = \overline x_2$} \\
\hline
$f_{11}$ & 1 & 0 & 1 & 1 & \makecell{Импликация \\ $f = x_2 \rightarrow x_1$} \\
\hline
$f_{12}$ & 1 & 1 & 0 & 0 & \makecell{$f = \overline x_1$} \\
\hline
$f_{13}$ & 1 & 1 & 0 & 1 & \makecell{Импликация \\ $f = x_1 \rightarrow x_2$} \\
\hline
$f_{14}$ & 1 & 1 & 1 & 0 & \makecell{Штрих Шеффера (отр. конъюнкции) \\$f = x_1 | x_2 = \overline{x_1 \land x_2}$} \\
\hline
$f_{15}$ & 1 & 1 & 1 & 1 & \makecell{Тавтология} \\
\hline
\end{tabular}
\end{center}
Функции без названий либо имеют очевидные названия, либо ущербны по своей природе и не заслуживают его.

\subsection{Основные законы алгебры логики}
Основные законы алгебры логики касаются конъюнкции и дизъюнкции. Возможно, на экзамене их потребуется доказать. Делать это лучше всего с помощью таблцы истинности, поскольку на этом этапе все законы как бы не доказаны и опираться на них нельзя.
\begin{enumerate}
  \item Идемпотентность: $x \land x = x$ \hspace{2cm} $x \lor x = x$
  \item Коммутативность: $x \land y = x \land y$ \hspace{2cm} $x \lor y = y \lor x$
  \item Ассоциативность: \\ $x \land (y \land z) = (x \land y) \land z$ \hspace{1cm} $x \lor (y \lor z) = (x \lor y) \lor z$
  \item Поглощение: $x \lor (x \land y) = x \hspace{1cm} x \land (x \lor y) = x$
  \item Дистрибутивность \\ $x \land (y \lor z) = (x \land y) \lor (x \lor z) \hspace{1cm} x \land (y \lor z) = (x \land y) \lor (x \lor z)$
  \item Правила де Моргана \\
  $\overline{x \lor y} = \overline x \land \overline y \hspace{1cm} \overline{x \land y} = \overline x \lor \overline y$
  \item Свойства констант
  $$ 1 \land x = x \hspace{1cm} 1 \lor x = 1 $$
  $$ 0 \land x = 0 \hspace{1cm} 0 \lor x = x $$
  \item Закон исключения третьего и закон противоречия
  $$ x \lor \overline x = 1 \hspace{1cm} x \land \overline x = 0$$
  \item Снятие двойного отрицания $\overline{\overline{x}} = x$
  \item Склеивание
  $$ (x \land y) \lor (\overline x \land y) = y$$
  $$ (x \lor y) \land (\overline x \lor y) = y$$
  \item Связь импликации с конъюнкцией, дизъюнкцией
  $$ x \rightarrow y = \overline{x} \lor y = \overline{x \land \overline{y}}$$
  \item Выражение эквивалентности
  $$ x \leftrightarrow y = (x \rightarrow y) \land (y \rightarrow x)$$
\end{enumerate}
Из правил де Моргана следует, что если все операции в тождестве --- это конъюнкции, дизъюнкции и отрицания, то при взаимной замене конъюнкций на дизъюнкции получится верное тождество.
\subsection{Следствия из законов алгебры логики: операции склеивания, поглощения, правила развертывания логических выражений}
См. выше. В этом вопросе, я полагаю, надо все это добро вывести. Также в списке вопросов дана ссылка на неясный источник
\subsection{Пять классов функций. Теорема о функциональной полноте}
Существуют следующие классы булевых функций:
\begin{itemize}
  \itembf{Самодвойственные ($S$).}
  \begin{defin}
  Булева функция $f^*$ называется \textbf{двойственной} к булевой функции $f$, если они обе принимают равное число $n$ аргументов и справедливо равенство:
  $$f^*(x_1, x_2, ..., x_n) = \overline{f(\overline x_1, \overline x_2, ..., \overline x_n)}$$
  \end{defin}

  {\small \textbf{Примечание.} Это равенство можно переписать как \newline
  $\overline{f^*(x_1, ..., x_n)} = f(\overline x_1, ..., \overline x_n)$}
  
  \begin{defin}
    Булева функция $f$ называется \textbf{самодвойственной}, если она является функцией, двойственной самой себе:
    $$ f^* = f $$
  \end{defin}

  \itembf{Монотонные ($M$)}
  \textit{Пусть} $ a = (a_1, ..., a_n) $ и $ b = (b_1, ..., b_n) $ --- наборы длины $n$ из 0 и 1.
  \begin{defin}
    Если справедливы неравенства\footnote{Договоримся, что $0 \le 0$, $1 \le 1$, $0 \le 1$.} $$a_1 \le b_1, ..., a_n \le b_n,$$
    то говорят, что набор $a$ \textbf{меньше} набора $b$ и пишут: $$ a \le b. $$
  \end{defin}
  \begin{defin}
  Если выполнено хотя бы одно из равенств: $a \le b$, $b \le a$, то наборы $a$ и $b$ \textbf{сравнимы}.
  \end{defin}
  \begin{defin}
    Булева функция $f$ называется монотонной, если для любых наборов $a$ и $b$ условие $a \le b$ влечет выполнение $f(a) \le f(b)$.
  \end{defin}

  \itembf{Линейные ($L$)}
  \begin{defin}
    \textbf{Полиномами Жегалкина} назваются формулы над множеством функций $F_J= \{0, 1, \land, \oplus\}$:
    \begin{align}
    P(x_1, x_2, &..., x_n) = a_0 \\
    \oplus a_1X_1 \oplus a_2&X_2 \oplus a_nX_n \\
    \oplus a_{12}X_{12} \oplus a_{13}X_1X_3 \oplus \cdots& \oplus a_{1\dots n}X_1 \dots X_n,
    \end{align}
    где $a_{i_1...i_n}$ --- постоянные члены, равные $0$ или $1$.
  \end{defin}

  Всякую булеву функцию можно представить единственным полиномом Жегалкина.

  \subparagraph{Алгоритм построения полинома Жегалкина по СДНФ.}\footnote{Не уверен, что это по смыслу это относится сюда, но в презентации оно дано}
    Пусть задана совершенная ДНФ функции $f(x_1, \dots, x_n)$.
    \begin{enumerate}
    \item Заменяем каждый символ дизъюнкции $\lor$ на символ дизюнкции с исключением $\oplus$.
    \item Заменяем каждую переменную с инверсией ($\overline x$) равносильной формулой $x \oplus 1$.
    \item Раскрываем скобки\footnote{Так, как если бы вместо конъюнкции было обычное умножение, а вместо xor'ов --- обычное сложение}.
    \item Вычеркиваем из формулы пары одинаковых слагаемых\footnote{Если в формуле четное число однородных слагаемых, то все уходят. Если нечетное --- отсается одно}.
    \end{enumerate}
    Получен полином Жегалкина функции $f(x_1, \dots, x_n)$.

    \begin{defin}
      Функция $f(x_1, ..., x_n)$ \textbf{линейная}, если ее многочлен Жегалкина является линейным относительно всех переменных, то есть имеет следующий вид:
      $$ f(x_1, ..., x_n) = a_1x_1 + \cdots + a_nx_n + a_{n+1} $$
    \end{defin}

  \itembf{Сохраняющие 0 ($T_0$)}
  \begin{defin}
    Говорят, что булева функция \textbf{сохраняет 0}, если выполнено равенство:
    $$ f(0, 0, ..., 0) = 0 $$
  \end{defin}

  \itembf{Сохраняющие 1 ($T_1$)}
  \begin{defin}
    Говорят, что булева функция \textbf{сохраняет 1}, если выполнено равенство:
    $$ f(1, 1, ..., 1) = 1 $$
  \end{defin}
\end{itemize}

Хотя означенные множества и называются классами, принадлежность функции к одному из них не исключает принадлежности и к другим классам.

\begin{theorem}
  (Поста, о функциональной полноте) Система булевых функций F является полной тогда и только тогда, когда она целиком не принадлежит ни одному из замкнутых классов $S$, $M$, $L$, $T_{0}$, $T_{1}$.
\end{theorem}

\subsection{Основная функционально полная система логических связей}
Понятия не имею, чего она хочет в этом вопросе.

\subsection{Теорема Жегалкина. Алгебра Жегалкина. Функции Шеффера и Пирса.}
Про алгебру Жегалкина см. выше (вопрос 1.5)
\subsection{}
\end{document}